\documentclass{article}

    \usepackage{fullpage}
    \usepackage{graphicx}
    \usepackage{color}
    \usepackage{fancyhdr}
    \usepackage{url}
    \usepackage{amsmath,bm}
    \usepackage{amssymb}
    \usepackage{amsthm}
    \usepackage{amsfonts}
    \usepackage[round]{natbib}
    \usepackage{enumitem,xcolor}
    \usepackage[multiple]{footmisc}
    
    \usepackage[
     pdftitle={Udacity Machine Learning Nanodegree: Capstone Proposal},
     pdfsubject={Machine Learning, Reinforcement Learning, Deep Learning, E-Coach, Food, Meal},
     pdfauthor={Marcio Nicolau},
     pdfpagemode=UseOutlines,
     pdfborder= {0 0 1.0},
     bookmarks,
     bookmarksopen,
     colorlinks=true,
     citecolor=blue,
     linkcolor=blue, %
     linkbordercolor=blue, %
     urlcolor=blue, %
    ]{hyperref}
    
    \usepackage[labelfont=bf]{caption}
    \usepackage[utf8]{inputenc}
    
    % Default fixed font does not support bold face
    \DeclareFixedFont{\ttb}{T1}{txtt}{bx}{n}{8} % for bold
    \DeclareFixedFont{\ttm}{T1}{txtt}{m}{n}{8}  % for normal
    
    % Custom colors
    \usepackage{color}
    \definecolor{deepblue}{rgb}{0,0,0.5}
    \definecolor{deepred}{rgb}{0.6,0,0}
    \definecolor{deepgreen}{rgb}{0,0.5,0}
    
    \usepackage{listings}
    
    \definecolor{codebg}{RGB}{238,238,238}
    
    % Python style for highlighting
    \newcommand\pythonstyle{\lstset{
    language=Python,
    basicstyle=\ttm,
    otherkeywords={self},             % Add keywords here
    keywordstyle=\ttb\color{deepblue},
    emph={MyClass,__init__},          % Custom highlighting
    emphstyle=\ttb\color{deepred},    % Custom highlighting style
    stringstyle=\color{deepgreen},
    frame=tb,                         % Any extra options here
    framesep=10pt,
    framexleftmargin=10pt,
    backgroundcolor=\color{codebg},
    rulecolor=\color{codebg},
    aboveskip=15pt,
    belowskip=15pt,
    showstringspaces=false            % 
    }}
    
    
    % Python environment
    \lstnewenvironment{python}[1][]
    {
    \pythonstyle
    \lstset{#1}
    }
    {}
    
    % Python for external files
    \newcommand\pythonexternal[2][]{{
    \pythonstyle
    \lstinputlisting[#1]{#2}}}
    
    % Python for inline
    \newcommand\pythoninline[1]{{\pythonstyle\lstinline!#1!}}
    
    \title{Udacity Machine Learning Nanodegree \\ Capstone Project Proposal}
    \author{Marcio Nicolau}
    \date{October 11, 2018}
    \begin{document}
\maketitle

\section{Domain Background}

In general, machine learning can be divided into three large areas designated as supervised, 
unsupervised and reinforcement learning, each one is recommended to solve a specific kind of problem, 
and for more complex tasks as learning (agent) by interaction (environment) and try to understand how 
to get the maximum overall reward during this process. Among all supervised learning techniques, one which 
gains a considerable attention is Deep Learning, that has been responsible for a dramatical improvement 
of state-of-the-art applications for areas from speech recognition to natural language process. 
The deep learning approach to reinforcement learning leads to an improvement of the problems complexity 
that Markov Decision Process (MDP) could solve.

Some of these recent achievements are extensively noticed, like the DeepMind AlphaGo and OpenAI Dota2. 
A good source of information related to this reinforcement learning context and applications could be 
found in the Sutton and Barto book ``Reinforcement Learning: An introduction'' (2th Ed., 2017). 
Alongside the game’s achievements, reinforcement learning is also widely used in the robotics field, 
to train robots to accomplish tasks without human interaction or domain-specific knowledge.

My personal motivation to work with reinforcement learning is related to better understand the 
process and application of this area of machine learning to a real-world problem. This is the 
closest implementation available today and related to the Artificial General Intelligence (AGI) concepts.

\section{Problem Statement}

In general, a person who want to loose weight or pretend to follow a well-defined and health meal plan, 
certainly well make an appointment with a nutritionist or dietician. This professional works the 
meals planning a recommendation for this individual, after collect some basic information about 
routines, food intake habits, goals, sleep pattern, activities (work, gym, etc.) and food 
preferences and restrictions. This approach works very well and for a long time, but, in some occasions, 
the person who have been following the meal plan need to make a change or adjust, based on eventual 
situations and this could cause a problem.

This proposal intends to develop a personal coach (e-coach) that in association with the professional 
assistance could help anyone to choose a food for a meal, and a deep reinforcement learning system 
shall suggest the next meal or food composition based on the optimal policy. The training procedure 
will consider the nominal intake calories daily\footnote{\url{https://www.calculator.net/calorie-calculator.html}}  
(based on the Mifflin-St Jeor Equation for man [Eq.~\ref{msj-man}] and for woman [Eq.~\ref{msj-woman}]) as the 
main goal and also, use the macronutrients\footnote{\url{https://www.bodybuilding.com/fun/macronutrients_calculator.htm}}  
proportion to help this person to have a better food intake behavior.

The value obtained from Eq.~\ref{msj-man} and Eq.~\ref{msj-woman} is the value of the basal metabolic rate (BMR)
and this is the estimated number of calories a person can consume in a day to maintain their body-weight assuming they 
remain at rest. This value is multiplied by an activity factor (generally between 1.2 and 1.95).

\begin{eqnarray}
    BMR = 10 \times weight\mbox{(kg)} + 6.25 \times height\mbox{(cm)} - 5 \times age\mbox{(years)} + 5 \label{msj-man}\\
    BMR = 10 \times weight\mbox{(kg)} + 6.25 \times height\mbox{(cm)} - 5 \times age\mbox{(years)} -161 \label{msj-woman}
\end{eqnarray}

For example, for a woman with 60 kg (weight), 168 cm (height), 30-year-old and moderate activity (1.55) the BMR and 
Total intake calories is:

\begin{itemize}
    \item BMR = 1,505
    \item Total Calories = 2,332 / day
\end{itemize}

\section{Datasets and Inputs}

This project will use a subset from ``USDA National Nutrient Dataset''\footnote{\url{https://ndb.nal.usda.gov/ndb/search/}}
with nutritional information (Calories, Protein, Total Fat, Carbohydrate, Fiber) 
for each food item used for the meal planning. Also, the Kaggle Open Food Fact dataset\footnote{\url{https://www.kaggle.com/openfoodfacts/world-food-facts/home}}  
will be used as complementary source of information to group food based on their 
nutritional information\footnote{\url{https://www.kaggle.com/lwodarzek/nutrition-table-clustering}}. 

From USDA National Nutrient Dataset, I choosed a subset (5662 items) with these content (SR-Legacy, 2018):

\begin{itemize}
    \item Dairy and Egg Products (291); Fats and Oils (216);
    \item Soups, Sauces, and Gravies (254); Sausages and Luncheon Meats (167);
    \item Breakfast Cereals (195); Fruits and Fruit Juices (355);
    \item Pork Products (336); Vegetables and Vegetable Products (814);
    \item Nut and Seed Products (137); Beef Products (954);
    \item Legumes and Legume Products (290); Baked Products (517);
    \item Sweets (358); Cereal Grains and Pasta (181);
    \item Fast Foods (312); Snacks (176); Restaurant Foods (109);
\end{itemize}

\section{Solution Statement}

A Deep Q-Learning algorithm will be used for generating the optimal policy. The state-action pair 
will be composed by amount of calories intake by a specific food or meal (Actions). The reward from 
environment will the sum of calories for a specific day, also an extension could be set the total 
reward for a hole week instead of only day achievement.

\section{Benchmark Model}

As benchmark, I propose a uniformly random agent and compare both using the average cumulative 
reward collected over the 50,000 runs for each agent.

\section{Evaluation Metrics}

\begin{itemize}
    \item Average cumulative reward, for a high number of simulations (e.g. 100,000) for 
    some combinations of Age, Sex, Stature (cm), Actual Weight (kg), Activity level 
    (sedentary, low, medium, intense) which will be used for calories intake goal 
    (Mifflin-St Jeor Equation).
    \item Root Mean Square Error (RMSE), for calories intake goal and cumulative 
    reward by combination of factors.
\end{itemize}

\section{Project Design}

\begin{itemize}
    \item Python 3
    \item Scikit-learn
    \item Pandas
    \item Numpy
    \item Keras
    \item Tensorflow
\end{itemize}

The Agent and Environment will be coded as Python Classes alongside with the data from 
USDA National Nutrient Dataset that will be pre-processed and used for collecting and 
extract nutritional information for each food and accessible for both Agent and 
Environment as a dictionary. 

The information about macronutrients (Protein, Total Fat, Carbohydrate) will be 
used as components for the reward functions with respective penalizations in 
association with the Calories (Energy) information for each food component.

\begin{python}
	class Game(object):
	def copy(self):
	'''Returns a copy of the game'''
	def getBoard(self):
	'''Returns the board of the game'''
	def getCurrentPlayer(self):
	'''Returns the player whose turn it is'''
	def getLegalMoves(self):
	'''Returns a list of legal moves for the player in turn'''
	def getOutcomes(self):
	'''Returns the outcome for each player at the end of the game'''
	def isGameOver(self):
	'''Returns true if the game is over, false otherwise'''
	def makeMove(self, move):
	'''Makes a move for the player whose turn it is'''
\end{python}

\bibliographystyle{plainnat}
\bibliography{bibliography}

\end{document}
